\documentclass[12 pt]{article}
\usepackage{easycalc3}
\usepackage{setspace}
\usepackage{enumerate}
\usepackage{lastpage}
\usepackage{float}
\usepackage{fancyhdr}
\usepackage{tikz}
\usepackage{tabularx}
\usepackage{ltablex}
\usepackage{textcomp}
\usepackage[T1]{fontenc}
\usepackage{listings}
\usepackage[margin=1 in]{geometry}
\allowdisplaybreaks
%\usepackage[dvipsnames]{xcolor}   %May be necessary if you want to color links
\usepackage{graphicx}
\graphicspath{{Images/}}
\author{Sarah Randall}
\date{Last updated: \today}
\title{MATH 222: Week 4}
\pagestyle{fancy}
\lhead{MATH 222}
\chead{\leftmark}
\rhead{Sarah Randall}
\cfoot{Page \thepage \ of \pageref{LastPage}}
\newcommand{\tab}[1]{\hspace{.2\textwidth}\rlap{#1}}
\begin{document}
    \onehalfspacing
    \maketitle
    \tableofcontents
    \section{\S 14.5 Chain Rule}
        The chain rule in 1-dimension is as follows:

        For an equation $y=f(x(t))$
        $$\frac{dy}{dt}=\frac{df}{dx}\frac{dx}{dt}$$
        \begin{exmp*}
            If $y=(x(t))^2$ and $x(t)=\ln{1+t}$\\
            $$\frac{dy}{dt}=\frac{dy}{dx}\frac{dx}{dt}=2x\frac{1}{1+t}=\frac{2\ln{1+t}}{1+t}$$
        \end{exmp*}

        \begin{exmp*}
            Suppose $f(x,y)=xy+x^2+y$\\
            $x(t)=\ln{1+t},\ y(t)=e^{t^2}$\\
            Turn $f(x,y)$ into a function $g(t)$ with only the time parameter.
            $$g(t)=f(x(t),y(t))=\ln{1+t}e^{t^2}+(\ln{1+t})^2+e^{t^2}$$
            $$\frac{dg}{dt}=\frac{df}{dt}=\frac{e^{t^2}}{1+t}+2t\ln{1+t}e^{t^2}+\frac{2\ln{1+t}}{1+t}+2te^{t^2}$$
            Wherever we can, replace the values of $x(t),\ y(t)$ with $x(t),\ y(t)$.
            \begin{align*}
                \frac{dg}{dt}=\frac{df}{dt}&=\frac{y(t)}{1+t}+2te^{t^2}x(t)+\frac{2}{1+t}+2te^{t^2}\\
                &=x'(t)y(t)+y'(t)x(t)+2x'(t)x(t)+y'(t)\\
                &=x'(t)(y(t)+2x(t))+y'(t)(x(t)+1)\\
                &=\frac{\partial f}{\partial x}\frac{dx}{dt}+\frac{\partial f}{\partial y}\frac{dy}{dt}
            \end{align*}
            In the second to last line, we use the fact that differentiating $f$ with respect to $x$ gives $y+2x$ and doing the same for $y$ gives $x+1$.
        \end{exmp*}

        There are two possible cases for the chain rule. Suppose in both cases we have $z=f(x,y)$. In the first case we have $x=g(t)$ and $y=h(t)$. In this case $z$ is a differentiable function of $t$.
        $$\frac{dz}{dt}=\frac{\partial f}{\partial x}\frac{dx}{dt}+\frac{\partial f}{\partial y}\frac{dy}{dt}$$
        In the second case, $x=g(s,t)$ and $y=h(s,t)$. Then $z$ is a differentiable function of both $s$ and $t$.
        $$\frac{dz}{ds}=\frac{\partial f}{\partial x}\frac{dx}{ds}+\frac{\partial f}{\partial y}\frac{dy}{ds}$$
        $$\frac{dz}{dt}=\frac{\partial f}{\partial x}\frac{dx}{dt}+\frac{\partial f}{\partial y}\frac{dy}{dt}$$

        \begin{exmp*}
            $z=e^x\cos(x+y),\ x=s^2t,\ y=st^2$. Find $\frac{dz}{ds}$ and $\frac{dz}{dt}$.
            $$\frac{\partial z}{\partial x}=e^x(\cos(x+y)-\sin(x+y))$$
            $$\frac{\partial z}{\partial y}=-e^x\sin(x+y)$$
            Then we need $\frac{dx}{ds},\ \frac{dx}{dt},\ \frac{dy}{ds},\ \frac{dy}{dt}$
            \begin{center}
                $$\frac{dx}{ds}=2st,\ \frac{dx}{dt}=s^2$$
                $$\frac{dy}{ds}=t^2,\ \frac{dy}{dt}=2st$$
            \end{center}
            So now we can find the general equations for $\frac{dz}{ds}$ and $\frac{dz}{dt}$.
            $$\frac{dz}{ds}=(e^x)(\cos(x+y)-\sin(x+y))(2st)+(-e^x)(\sin(x+y))(t^2)$$
            $$\frac{dz}{dt}=(e^x)(\cos(x+y)-\sin(x+y))(s^2)+(-e^x)(\sin(x+y))(2st)$$
        \end{exmp*}

        \begin{exmp*}
            If $g(s,t)=f(s^2-t^2,t^2-s^2)$ and $f$ is differentiable, show that $t\frac{dg}{ds}+s\frac{dg}{dt}=0$.\\\\
            Based on $f$, $x(s,t)=s^2-t^2$ and $y(s,t)=t^2-s^2$. Therefore we can write that $g(s,t)=f(x(s,t),y(s,t))$. We need to find $\frac{dg}{dt}$ and $\frac{dg}{ds}$ and to do this we need need to differentiate $x(s,t),\ y(s,t)$ by both $s$ and $t$.
            $$\frac{dg}{dt}=\frac{\partial f}{\partial x}\frac{dx}{dt}+\frac{\partial f}{\partial y}\frac{dy}{dt}=\frac{\partial f}{\partial x}(-2t)+\frac{\partial f}{\partial y}(2t)$$
            $$\frac{dg}{ds}=\frac{\partial f}{\partial x}\frac{dx}{ds}+\frac{\partial f}{\partial y}\frac{dy}{ds}=\frac{\partial f}{\partial x}(2s)+\frac{\partial f}{\partial y}(-2s)$$
            If we multiply the first equation all by $t$ and the second all by $s$, we get
            $$t\frac{dg}{ds}=-2st\frac{\partial f}{\partial x}+2st\frac{\partial f}{\partial y}$$
            $$s\frac{dg}{dt}=2st\frac{\partial f}{\partial x}-2st\frac{\partial f}{\partial y}$$
            Doing linear combination gives
            $$t\frac{dg}{ds}+s\frac{dg}{dt}=0$$
        \end{exmp*}

        \subsection{Chain rule and implicit functions}

        In 1 dimension, if we had an implicit function $F(x,y)=0$ we would do the following to differentiate it.
        $$\frac{dF}{dx}(x,y)=0\ \Rightarrow\ \frac{dF}{dx}\frac{dx}{dx}+\frac{dF}{dy}\frac{dy}{dx}=0$$
        $\frac{dx}{dx}$ always equals 1, so we get an equation for $\frac{dy}{dx}$.
        $$\frac{dy}{dx}=\frac{\frac{-dF}{dx}}{\frac{dF}{dy}}$$
        Provided that $\frac{dF}{dy}$ doesn't equal 0.

        We can apply this idea to an implicit function like $F(x,y,x)=0$ as well
        $$\frac{dF}{dx}=\frac{dF}{dx}\frac{dx}{dx}+\frac{dF}{dy}\frac{dy}{dx}+\frac{dF}{dz}\frac{dz}{dx}=0$$
        Like before, $\frac{dx}{dx}$ is 1. In addition, because $y$ is no longer a function of $x$ we can say that $\frac{dy}{dx}=0$. $y$ doesn't depend on $x$ at all. However, $z$ does depend on $x$ so that stays put.
        $$0=\frac{dF}{dx}+\frac{dF}{dz}\frac{dz}{dx}$$
        $$\frac{dz}{dx}=\frac{\frac{-dF}{dx}}{\frac{dF}{dz}}$$

        \subsection{Implicit function theorem}

        Suppose a function $F(x,y,z)$ is defined on a sphere around a point $(a,b,c)\in\R^3$ satisfying $F(a,b,c)=0$.

        If $F_x,\ F_y,\ F_z$ are continuous and $\frac{dF}{dz}$ evaluated at $(a,b,c)$ does not equal 0, then in a neighborhood of $(a,b,c)$ we have that the equation $F(x,y,z)=0$ defines $z$ as a function of $x,\ y$ near $(a,b,c)$ in this neighborhood. In addition, this function is differentiable in this area and its partial derivatives $\frac{dz}{dx}=f_x(x,y),\ \frac{dz}{dy}=f_y(x,)$ are given by
        $$\frac{dz}{dx}=\frac{\frac{-dF}{dx}}{\frac{dF}{dz}}$$
        $$\frac{dz}{dy}=\frac{\frac{-dF}{dy}}{\frac{dF}{dz}}$$

        This theorem is really a test that we use to determine if we can get a tangent plane at a given point on the function. If the slope is $\infty$ then our equations for $\frac{dz}{dx},\ \frac{dz}{dy}$ won't work because $\frac{dF}{dz}$ will be zero.

        In clas we used the example of a simple circle $x^2+y^2=1$, a circle of radius 1 centered at the origin. If we choose $(a,b)$ to be somewhere in the middle of the top-right quadrant, in this neighborhood of $(a,b)$ we can talk about the curve as a function. There is no point in this neighborhood where the slope is $\infty$. However, if we choose $(a,b)$ to be $(1,0)$ or $(-1,0)$ we have a situation where the curve is not a function. We can tell this by looking at the graph of the circle but for more confusing curves we need the Implicit Function Theorem. If $f(a,b)$ has a slope of $\infty$ then since we're going to be dealing with the neighborhood of $(a,b)$, there will be points in this neighborhood that fail the vertical line test. Then we can't find derivatives and tangent planes in this area since it's not a function here.

        \begin{exmp*}
            Let $F(x,y,z)=0$ for $F(x,y,z)=x^3+y^3+z^3+6zyz-9$. Show that around $(1,1,1)$ we can define $z$ as a function of $x,\ y$. Find the values of $\frac{dz}{dy}$ and $\frac{dz}{dx}$ at $(1,1,1)$.

            First check if the point is on the surface. $1+1+1+6-9=0$, so it's on the surface. Next we need to check that $\frac{dF}{dz}$ at the point isn't 0.
            $$\frac{dF}{dz}=3z^2+6xy$$
            Evaluating this is the point gives 9, which isn't 0. Since $F$ is a polynomial we can also say that $F_x,F_y,F_z$ are continuous .

            By the Implicit Function Theorem, near $(1,1,1)$ $z$ is a function of $x,\ y$. Now we need to find $\frac{dz}{dy}$ and $\frac{dz}{dx}$.
            $$\frac{dz}{dx}=\frac{-F_x}{F_z}=\frac{-(3x^2+6yz)}{3z^2+6xy}\Big|_{1,1,1}=-1$$
            $$\frac{dz}{dy}=\frac{-F_y}{F_z}=\frac{-(3y^2+6xz)}{3z^2+6xy}\Big|_{1,1,1}=-1$$
        \end{exmp*}

        \subsection{Directional Derivative}

        \begin{def*}
            The directional derivative of $f(x,y)$ at $(x_0,y_0)$ in the direction of a unit vector $\vec{u}=<a,b>$ is
            $$D_{\vec{u}}f(x_0,y_0)=\lim_{h\rightarrow 0}\frac{f(x_0+ah,y_0+bh)-f(x_0,y_0)}{h}$$
        \end{def*}
        Let $g(h)=f(x+ah, y+bh)$ so that $g(0)=f(x,y)$.
        $$g'(0)=\lim_{h\rightarrow 0}\frac{g(h)-g(0)}{h}=D_{\vec{u}}f(x,y)$$
        We can rewrite this using what we know about the chain rule
        $$g'(h)=\frac{\partial f}{\partial x}\frac{dx}{dh}+\frac{\partial f}{\partial y}\frac{dy}{dh}$$
        We can say that $x(h)=x_0+ah$ and $y(h)=y_0+bh$. From this we can also say that $\frac{dx}{dh}=a$ and $\frac{dy}{dh}=b$. Substitute these into this equation.
        $$g'(h)=af_x(x+ah,y+bh)+bf_y(x+ah,y+bh)$$
        Set $h=0$
        $$g'(0)=af_x(x,y)+bf_y(x,y)$$
        Since we had previously that $g'(0)=D_{\vec{u}}f(x,y)$, we can set these equal and get
        $$D_{\vec{u}}f(x,y)=af_x(x,y)+bf_y(x,y)$$
        Another common way of writing this is
        $$D_{\vec{u}}f(x,y)=<a,b>\cdot<f_x(x,y),f_y(x,y)>$$
        $D_{\vec{u}}f$ is the rate of change of $f$ in direction $\vec{u}$. $\vec{u}$ must be a unit vector because we use its coordinates in our equation. If $\vec{u}$ isn't a unit vector, we won't get the correct answer.

        \begin{exmp*}
            Compute the directional derivative of $f(x,y)=x^2+2y^2+y$ at $(1,1)$ in direction $<1,2>$\\\\
            To find $a,\ b$ we need to divide the coordinates of $\vec{u}$ by the vector's length.
            $$a=\frac{1}{\magn{u}}=\frac{\sqrt{5}}{5}$$
            $$b=\frac{2}{\magn{u}}=\frac{2\sqrt{5}}{5}$$
            Now if we just find $f_x$ and $f_y$, we can find the directional derivative.
            $$D_{\vec{u}}=<\frac{\sqrt{5}}{5},\frac{2\sqrt{5}}{5}>\cdot<2x,4y+1>$$
            Plug in the point we were given.
            $$<\frac{\sqrt{5}}{5},\frac{2\sqrt{5}}{5}>\cdot<2,5>=\frac{12\sqrt{5}}{5}$$
            This is the directional derivative at $(1,1)$ in direction $<1,2>$.
        \end{exmp*}

        \subsection{The Gradient Vector}

        A question that naturally follows from this is for what $\vec{u}=<a,b>$ do we get the largest directional derivative? If we want to maximize $<a,b>\cdot<f_x(x,y),f_y(x,y)>$ we should look at an identity we learned earlier in the course.
        $$<a,b>\cdot<f_x(x,y),f_y(x,y)>=\parallel<a,b>\parallel\ \parallel<f_x,f_y>\parallel\cos(\theta)$$
        We already know that the length of $<a,b>$ is 1. $<f_x,f_y>$ has a fixed sized. $\cos(\theta)$ is largest at $\theta=0$. Therefore the directional derivative is largest when $<a,b>$ is parallel to $f_x,f_y>$ (when the angle between them is 0). Therefore $D_{\vec{u}}f$ is maximized when $<a,b>=\frac{<f_x,f_y>}{\parallel<f_x,f_y>\parallel}$.

        We call this vector in the direction of the maximum change if you start at some point $x_0,y_0$ the gradient vector:
        $$\nabla f(x_0,y_0)=<f_x(x_0,y_0),f_y(x_0,y_0)>$$
        We don't necessarily need this to be a unit vector, so we won't divide the vector by its length here like it would be in the directional derivative.

        \begin{exmp*}
            Find the direction of the maximum derivative for $f(x,y)=x^2+2y^2+y$ and find the value of that max derivative.\\\\
            The direction of the max derivative will be the gradient vector. Find $f_x,\ f_y$ and plug them into our gradient vector equation.
            $$\nabla f=<2x,4y+1>\ \Rightarrow\ \nabla f(1,1)=<2,5>$$
            This vector gives the direction (we don't really care that it's not a unit vector). However, to find the max value of the directional derivative we'll need to do:
            $$D_{\vec{u}}f(x,y)=<a,b>\cdot<f_x(x,y),f_y(x,y)>$$
            In this case, $<a,b>$ is $<\frac{2}{\sqrt{29}},\frac{5}{\sqrt{29}}>$ since we want the directional derivative in this direction. We already know that $<f_x,f_y>$ is $<2,5>$ because we already calculated these values.
            $$D_{\nabla f}f(1,1)=\frac{<2,5>}{\parallel <2,5>\parallel}\cdot<2,5>=\frac{\parallel<2,5>\parallel^2}{\parallel<2,5>\parallel}=\sqrt{29}$$
        \end{exmp*}
        This gives us valuable information. The maximum directional derivative (directional derivative calculated at $\nabla f$) will be
        $$D_{\nabla f}f=\frac{<f_x,f_y>}{\parallel <f_x,f_y>\parallel}\cdot<f_x,f_y>=\frac{\parallel<f_x,f_y>\parallel^2}{\parallel<f_x,f_y>\parallel}=\parallel<f_x,f_y>\parallel$$
        We can do this because of the rule we learned early in the course that the dot product of a vector and itself equals the magnitude squared of that vector.

        Here is an example of a type of problem that might make more sense. As a brief note, the directional derivative in 3D is as follows:
        $$D_{\vec{u}}f=<u_1,u_2,u_3>\cdot\nabla f(x,y,z)$$
        \begin{exmp*}
            Suppose the temperature at a point $(x,y,z)$ in space is given by $T(x,y,z)=\frac{80}{1+x^2+2y^2+3z^2}$ in degrees Celsius. In what direction does the temperature increase fastest when starting from $(1,1,-2)$? What is the rate of increase?\\\\
        \end{exmp*}
    \section{\S 14.7 Maximum and Minimum Values}
    \section{\S 14.8 Lagrange Multipliers}
    %I'm going to keep the beginning of chapter 15 out of this document to keep things organized
\end{document}
