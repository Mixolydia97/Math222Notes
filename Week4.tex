\documentclass[12 pt]{article}
\usepackage{easycalc3}
\usepackage{setspace}
\usepackage{enumerate}
\usepackage{lastpage}
\usepackage{float}
\usepackage{fancyhdr}
\usepackage{tikz}
\usepackage{tabularx}
\usepackage{ltablex}
\usepackage{textcomp}
\usepackage[T1]{fontenc}
\usepackage{listings}
\usepackage[margin=1 in]{geometry}
\allowdisplaybreaks
%\usepackage[dvipsnames]{xcolor}   %May be necessary if you want to color links
\usepackage{graphicx}
\graphicspath{{Images/}}
\author{Sarah Randall}
\date{Last updated: \today}
\title{MATH 222: Week 4}
\pagestyle{fancy}
\lhead{MATH 222}
\chead{\leftmark}
\rhead{Sarah Randall}
\cfoot{Page \thepage \ of \pageref{LastPage}}
\newcommand{\tab}[1]{\hspace{.2\textwidth}\rlap{#1}}
\begin{document}
    \onehalfspacing
    \maketitle
    \tableofcontents
    \section{\S 14.5 Chain Rule}
        The chain rule in 1-dimension is as follows:

        For an equation $y=f(x(t))$
        $$\frac{dy}{dt}=\frac{df}{dx}\frac{dx}{dt}$$
        \begin{exmp*}
            If $y=(x(t))^2$ and $x(t)=\ln{1+t}$\\
            $$\frac{dy}{dt}=\frac{dy}{dx}\frac{dx}{dt}=2x\frac{1}{1+t}=\frac{2\ln{1+t}}{1+t}$$
        \end{exmp*}

        \begin{exmp*}
            Suppose $f(x,y)=xy+x^2+y$\\
            $x(t)=\ln{1+t},\ y(t)=e^{t^2}$\\
            Turn $f(x,y)$ into a function $g(t)$ with only the time parameter.
            $$g(t)=f(x(t),y(t))=\ln{1+t}e^{t^2}+(\ln{1+t})^2+e^{t^2}$$
            $$\frac{dg}{dt}=\frac{df}{dt}=\frac{e^{t^2}}{1+t}+2t\ln{1+t}e^{t^2}+\frac{2\ln{1+t}}{1+t}+2te^{t^2}$$
            Wherever we can, replace the values of $x(t),\ y(t)$ with $x(t),\ y(t)$.
            \begin{align*}
                \frac{dg}{dt}=\frac{df}{dt}&=\frac{y(t)}{1+t}+2te^{t^2}x(t)+\frac{2}{1+t}+2te^{t^2}\\
                &=x'(t)y(t)+y'(t)x(t)+2x'(t)x(t)+y'(t)\\
                &=x'(t)(y(t)+2x(t))+y'(t)(x(t)+1)\\
                &=\frac{\partial f}{\partial x}\frac{dx}{dt}+\frac{\partial f}{\partial y}\frac{dy}{dt}
            \end{align*}
            In the second to last line, we use the fact that differentiating $f$ with respect to $x$ gives $y+2x$ and doing the same for $y$ gives $x+1$.
        \end{exmp*}

        There are two possible cases for the chain rule. Suppose in both cases we have $z=f(x,y)$. In the first case we have $x=g(t)$ and $y=h(t)$. In this case $z$ is a differentiable function of $t$.
        $$\frac{dz}{dt}=\frac{\partial f}{\partial x}\frac{dx}{dt}+\frac{\partial f}{\partial y}\frac{dy}{dt}$$
        In the second case, $x=g(s,t)$ and $y=h(s,t)$. Then $z$ is a differentiable function of both $s$ and $t$.
        $$\frac{dz}{ds}=\frac{\partial f}{\partial x}\frac{dx}{ds}+\frac{\partial f}{\partial y}\frac{dy}{ds}$$
        $$\frac{dz}{dt}=\frac{\partial f}{\partial x}\frac{dx}{dt}+\frac{\partial f}{\partial y}\frac{dy}{dt}$$

        \begin{exmp*}
            $z=e^x\cos(x+y),\ x=s^2t,\ y=st^2$. Find $\frac{dz}{ds}$ and $\frac{dz}{dt}$.
            $$\frac{\partial z}{\partial x}=e^x(\cos(x+y)-\sin(x+y))$$
            $$\frac{\partial z}{\partial y}=-e^x\sin(x+y)$$
            Then we need $\frac{dx}{ds},\ \frac{dx}{dt},\ \frac{dy}{ds},\ \frac{dy}{dt}$
            \begin{center}
                $$\frac{dx}{ds}=2st,\ \frac{dx}{dt}=s^2$$
                $$\frac{dy}{ds}=t^2,\ \frac{dy}{dt}=2st$$
            \end{center}
            So now we can find the general equations for $\frac{dz}{ds}$ and $\frac{dz}{dt}$.
            $$\frac{dz}{ds}=(e^x)(\cos(x+y)-\sin(x+y))(2st)+(-e^x)(\sin(x+y))(t^2)$$
            $$\frac{dz}{dt}=(e^x)(\cos(x+y)-\sin(x+y))(s^2)+(-e^x)(\sin(x+y))(2st)$$
        \end{exmp*}

        \begin{exmp*}
            If $g(s,t)=f(s^2-t^2,t^2-s^2)$ and $f$ is differentiable, show that $t\frac{dg}{ds}+s\frac{dg}{dt}=0$.\\\\
            Based on $f$, $x(s,t)=s^2-t^2$ and $y(s,t)=t^2-s^2$. Therefore we can write that $g(s,t)=f(x(s,t),y(s,t))$. We need to find $\frac{dg}{dt}$ and $\frac{dg}{ds}$ and to do this we need need to differentiate $x(s,t),\ y(s,t)$ by both $s$ and $t$.
            $$\frac{dg}{dt}=\frac{\partial f}{\partial x}\frac{dx}{dt}+\frac{\partial f}{\partial y}\frac{dy}{dt}=\frac{\partial f}{\partial x}(-2t)+\frac{\partial f}{\partial y}(2t)$$
            $$\frac{dg}{ds}=\frac{\partial f}{\partial x}\frac{dx}{ds}+\frac{\partial f}{\partial y}\frac{dy}{ds}=\frac{\partial f}{\partial x}(2s)+\frac{\partial f}{\partial y}(-2s)$$
            If we multiply the first equation all by $t$ and the second all by $s$, we get
            $$t\frac{dg}{ds}=-2st\frac{\partial f}{\partial x}+2st\frac{\partial f}{\partial y}$$
            $$s\frac{dg}{dt}=2st\frac{\partial f}{\partial x}-2st\frac{\partial f}{\partial y}$$
            Doing linear combination gives
            $$t\frac{dg}{ds}+s\frac{dg}{dt}=0$$
        \end{exmp*}

        \subsection{Chain rule and implicit functions}

        In 1 dimension, if we had an implicit function $F(x,y)=0$ we would do the following to differentiate it.
        $$\frac{dF}{dx}(x,y)=0\ \Rightarrow\ \frac{dF}{dx}\frac{dx}{dx}+\frac{dF}{dy}\frac{dy}{dx}=0$$
        $\frac{dx}{dx}$ always equals 1, so we get an equation for $\frac{dy}{dx}$.
        $$\frac{dy}{dx}=\frac{\frac{-dF}{dx}}{\frac{dF}{dy}}$$
\end{document}
